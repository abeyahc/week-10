\documentclass{report}

\input{preamble}
\input{macros}
\input{letterfonts}

\usepackage{mathtools}

\title{\Huge{Discrete Mathematics}\\Week 8}
\author{\huge{Abeyah Calpatura}}
\date{}

\begin{document}
\maketitle
\section*{8.3}
\subsection*{Exercises} \\
\text{Abeyah Calpatura} \\
\#3,7, 15ab, 16a, 17,36, 37, 38 \\ \\

\noindent \textbf{\#3}
\sol{ $$A = \{0, 1, 2, 3, 4\}$$
$$R = \{(0, 0), (0, 4), (1, 1), (1, 3), (2, 2), (3, 1), (3, 3), (4, 0), (4, 4)\}$$
\vspace*{2cm}
\begin{align*}
    & \text{equivalence classes: [0], [1], [2], [3]} \\
    \vspace{3cm} \\
    & \text{$[0] = \{x \in A \mid x \: R \: 0 \} = \{0, 4\}$} \\ 
    & \text{$[1] = \{x \in A \mid x \: R \: 1 \} = \{1, 3\}$} \\ 
    & \text{$[2] = \{x \in A \mid x \: R \: 2 \} = \{2\}$} \\ 
    & \text{$[3] = \{x \in A \mid x \: R \: 3 \} = \{1, 3\}$} \\ 
    & \text{$[4] = \{x \in A \mid x \: R \: 4 \} = \{0, 4\}$} \\ \\
    & \text{The distinct equivalence classes of the relation R are $\{0, 4\}, \{2\}, \{1, 3\}$}
\end{align*}
}   

\noindent \textbf{\#7}
\sol{$A = \{(1, 3), (2, 4), (-4, -8), (3, 9), (1, 5), (3, 6)\}$ \\
R is defined on A as follows: For every (a, b), (c, d) $\in A$, 
$$(a, b) \: R \: (c, d) \iff ad=bc$$ \\
Find the distinct equivalence classes of the relation R. \\
\begin{align*}
    & \text{$[(1, 3)] = \{(a, b) \in A: (a, b) R (1, 3)\}$} \\ 
    & \text{$= \{(a, b) \in A: 3a=b\}$} \\
    & \text{$= \{(1, 3), (3, 9)\}$} \\ \\
    & \text{$[(2, 4)] = \{(a, b) \in A: (a, b) R (2, 4)\}$} \\
    & \text{$= \{(a, b) \in A: 4a=2b\}$} \\
    & \text{$= \{(2, 4), (-4, -8), (3, 6)\}$} \\ \\ 
    & \text{$[(1, 5)] = \{(a, b) \in A: (a, b) R (1, 5)\}$} \\
    & \text{$= \{(a, b) \in A: 5a=b\}$} \\
    & \text{$= \{(1, 5)\}$} \\
\end{align*}
}
\newpage
\noindent \textbf{\#15a}
\sol{ $17 \equiv 2 \pmod{5}$ \\
\begin{align*}
    &\text{$17-2=15=3\cdot5$} \\
    &\text{True}
\end{align*}
}
\textbf{\#15b}
\sol{$4 \equiv -5 \pmod{7}$ \\s
\begin{align*}
    &\text{$4-(-5)=4+5=9$} \\
    &\text{False}
\end{align*}
}

\noindent \textbf{\#16a}
\sol{ Let R be the relation of congruence modulo 3. Which of the following equivalence classes are equal?
$$[7], [-4], [-6], [17], [4], [27], [19]$$
$$R = \text{Relation of congruence modulo 3}$$
\begin{align*} 
    \begin{tabular}{ | c | c | c | c |}
    \hline
    Equivalence class & a - 7 & 3 divides a - 7 & Equal to equivalence class [7] \\
    \hline
    [-4] & $-4-7=-11$ & No & No \\
    \hline
    [-6] & $-6-7=-13$ & No & No \\
    \hline
    [17] & $17-7=10$ & No & No \\
    \hline
    [4] & $4-7=-3$ & Yes & Yes \\
    \hline
    [27] & $27-7=20$ & No & No \\
    \hline
    [19] & $19-7=12$ & Yes & Yes \\    
    \hline
    \end{tabular}
\end{align*}
\begin{align*}
    & \text{Implies: $[7] = [4] =[19]$}
\end{align*}
\begin{align*}
    \begin{tabular}{ | c | c| c | c |}
        \hline
        Equivalence class & $a - (-4)$ & 3 divides $a - (-4)$ & Equal to equivalence class [-4] \\
        \hline
        [-6] & $-6-(-4)=-2$ & No & No \\
        \hline
        [17] & $17-(-4)=21$ & Yes & Yes \\
        \hline
        [27] & $27-(-4)=31$ & No & No \\
        \hline
    \end{tabular}
\end{align*}
\begin{align*}
    & \text{Implies: $[-4] = [17]$}
\end{align*}

\begin{align*}
    \begin{tabular}{ | c | c| c | c |}
        \hline
        Equivalence class & $a - (-6)$ & 3 divides $a - (-6)$ & Equal to equivalence class [-6] \\
        \hline
        [27] & $27-(-6)=-33$ & Yes & Yes \\
        \hline
    \end{tabular}
\end{align*}
\begin{align*}
    & \text{Implies: $[-6] = [27]$}
\end{align*}
}

\noindent s\textbf{\#17a}
\sol{Prove that all integers m and n, $m \equiv n \pmod{3}$ if, and only if \textit{m mod 3 = n mod 3}
\begin{align*}
    & \text{Let m and n be integers} \\
    & \text{Let m mod 3 = n mod 3} \\
    & \text{Then, m = 3a + b and n = 3c + b, where a, b, c are integers and 0 $\leq$ b $<$ 3} \\
    & \text{Then, m - n = 3a + b - 3c - b} \\
    & \text{m - n = 3(a - c)} \\
    & \text{3 divides m - n} \\
    & \text{m $\equiv$ n $\pmod{3}$}
\end{align*}
}
\textbf{\#17b}
\sol{ Prove that for all integers m and n and any positive integer d, $m \equiv n \pmod{d}$ if, and only if, \textit{m mod d = n mod d}
\begin{align*}
    & \text{Let m and n be integers} \\
    & \text{Let m mod d = n mod d} \\
    & \text{Then, m = da + b and n = dc + b, where a, b, c are integers and 0 $\leq$ b $<$ d} \\
    & \text{Then, m - n = da + b - dc - b} \\
    & \text{m - n = d(a - c)} \\
    & \text{d divides m - n} \\
    & \text{m $\equiv$ n $\pmod{d}$}
\end{align*}

}

\noindent \textbf{\#36}
\sol{ For every a in A, $a \in A$
\begin{align*}
    & \text{Let $a \in A$} \\
    & \text{Since R is an equivalence relation, R is reflexive, symmetric, and transitive} \\
    & \text{By definition of reflexive: (a, a) $\in R$ or equivalently a R a} \\
    & \text{a R a is true and since $a \in A$, we note that $a \in [a]$} \\
\end{align*}
}

\noindent \textbf{\#37}
\sol{ For every a and b in A, if $b \in [a]$ then $a\:R\:b$.
\begin{align*}
    & \text{Let a and b be in A} \\
    & \text{Let $b \in [a]$} \\
    & \text{By definition of equivalence class: $b \in [a]$ if and only if $b \in A$ and $a\:R\:b$} \\
    & \text{Since $b \in [a]$, $b \in A$ and $a\:R\:b$} \\
    & \text{Therefore, if $b \in [a]$ then $a\:R\:b$ by using that R is symmetric and definition of the equivalence class}
\end{align*}
}

\noindent \textbf{\#38}
\sol{For every a, b, and c in A, if $b \: R \: c$ and $c \in [a]$ then $b\in [a]$.
\begin{align*}
    & \text{Let a, b, and c be in A} \\
    & \text{Let $b \: R \: c$ and $c \in [a]$} \\
    & \text{By definition of equivalence class: $c \in [a]$ if and only if $c \in A$ and $a\:R\:c$} \\
    & \text{Since $c \in [a]$, $c \in A$ and $a\:R\:c$} \\
    & \text{Since $b \: R \: c$, $b \: R \: c$ and $a\:R\:c$} \\
    & \text{Prove using that R is transitive and the definitino of equivalence class} \\
\end{align*}
}

\newpage

\section*{8.4}
\subsection*{Exercises} \\
\text{Abeyah Calpatura} \\
\#1, 3, 7, 14, 15,19, 22, 26, 31, 36, 39 \\ \\

\noindent \textbf{\#1a}
\sol{WHERE SHALL WE MEET
\begin{align*}
    & \text{23 08 05 18 05   19 08 01 12 12   23 05   13 05 05 20} \\
    & \text{$C = (M+3)$} \\
    & \text{26 11 08 21 08   22 11 04 15 15   26 08   16 08 08 23} \\
    & \text{ZKHUH VKDOO ZH PHHW}
\end{align*}
}
\textbf{\#1b}
\sol{ LQ WKH FDIHWHULD
\begin{align*}
    & \text{12 17   23 11 08   06 04 09 08 23 08 21 12 04} \\
    & \text{$C = (M-3)$} \\
    & \text{09 14 20   20 08 05   03 01 06 05 20 05 18 09 01} \\
    & \text{IN THE CAFETERIA}
\end{align*}
}

\noindent \textbf{\#3}
\text{Let a = 25, b = 19, and n = 3} \\ 
\textbf{\#3a}
\sol{ Verify that $3 \mid (25-19)$
\begin{align*}
    & \text{$25-19=6=3 \cdot 2$}
\end{align*}
}

\noindent \textbf{\#3b}
\sol{ Explain why $25 \equiv 19 \pmod{3}$
\begin{align*}
    & \text{Through part a, we determined that 3 divides $25-19$}
\end{align*}

}

\noindent \textbf{\#3c}
\sol{What value of k has the proprety that $25 = 19+3k$?
\begin{align*}
    & \text{$25=19+3k$} \\
    & \text{$6=3k$} \\
    & \text{$k=2$} \\
\end{align*}
}

\noindent \textbf{\#3d}
\sol{What is the (nonnegative) remainder when 25 is divided by 3? When 19 is divided by 3?
\begin{align*}
    & \text{$25 \div 3 = 8$ remainder 1} \\
    & \text{$19 \div 3 = 6$ remainder 1} \\
\end{align*}
}

\noindent \textbf{\#3e}
\sol{Explain why 25 mod 3 = 19 mod 3
\begin{align*}
    & \text{The remainder when 25 is divided by 3 is 1} \\
    & \text{The remainder when 19 is divided by 3 is 1} \\
    & \text{Both remainders are 1 }
\end{align*}
}

\noindent \textbf{\#7a}
\sol{$128 \equiv 2 \pmod{7} \: \text{and} \: 61 \equiv 5 \pmod{7}$
\begin{align*}
    & \text{$7 \mid (128-2)$} \\
    & \text{$128-2=126=7 \cdot 18$} \\
    & \text{$7 \mid (61-5)$} \\
    & \text{$61-5=56=7 \cdot 8$} \\
\end{align*}
}

\noindent \textbf{\#7b}
\sol{ $(128+61) \equiv (2+5)\pmod{7}$
\begin{align*}
    & \text{$128+61=189$} \\
    & \text{$2+5=7$} \\
    & \text{$7 \mid ((128+61)-(2+5))$} \\
    & \text{$7 \mid (189-7)$} \\
    & \text{$7 \mid 182$} \\
    & \text{$182=7 \cdot 26$} \\
\end{align*}
}

\noindent \textbf{\#7c}
\sol{ $(128-61) \equiv (2-5) \pmod{7}$
\begin{align*}
    & \text{$128-61=67$} \\
    & \text{$2-5=-3$} \\
    & \text{$7 \mid ((128-61)-(2-5))$} \\
    & \text{$7 \mid (67+3)$} \\
    & \text{$7 \mid 70$} \\
    & \text{$70=7 \cdot 10$} \\
\end{align*}
}

\noindent \textbf{\#7d}
\sol{ $(128 \cdot 61) \equiv (2 \cdot 5) \pmod{7}$
\begin{align*}
    & \text{$128 \cdot 61=7808$} \\
    & \text{$2 \cdot 5=10$} \\
    & \text{$7 \mid ((128 \cdot 61)-(2 \cdot 5))$} \\
    & \text{$7 \mid (7808-10)$} \\
    & \text{$7 \mid 7798$} \\
    & \text{$7798=7 \cdot 1114$} \\
\end{align*}
}

\noindent \textbf{\#7e}
\sol{ $128^2 \equiv 2^2 \pmod{7}$
\begin{align*}
    & \text{$128^2=16384$} \\
    & \text{$2^2=4$} \\
    & \text{$7 \mid (16384-4)$} \\
    & \text{$7 \mid 16380$} \\
    & \text{$16380=7 \cdot 2340$} \\
\end{align*}
}

\noindent \textbf{\#14}
\sol{Use the technique of Example 8.4.4 to find $14^2 \: mod \: 55$, $14^4 \: mod \: 55$, $14^8 \: mod \: 55$, $14^{16} \: mod \: 55$
\begin{align*}
    & \text{$14^2 \: mod \: 55 = 196 \: mod \: 55 = 31$} \\
    & \text{$14^4 \: mod \: 55 = (14^2 \: mod \: 55)^2 = (31)^2 \: mod \: 55 = 26$} \\
    & \text{$14^8 \: mod \: 55 = (14^4 \: mod \: 55)^2 = (26)^2 \: mod \: 55 = 16$} \\
    & \text{$14^{16} \: mod \: 55 = (14^8 \: mod \: 55)^2 = (16)^2 \: mod \: 55 = 36$} \\
\end{align*}
}

\noindent \textbf{\#15}
\sol{Use the result of \#14 to find $14^{27} \: mod \: 55$
\begin{align*}
    & \text{$14^{27} \: mod \: 55 = (14^{16} \cdot 14^{8} \cdot 14^2 \cdot 14^1) \: mod \: 55$} \\
    & \text{$14^{27} \: mod \: 55 = (31 \cdot 16 \cdot 26 \cdot 36) \: mod \: 55$} \\
    & \text{$14^{27} \: mod \: 55 = 249984 \: mod \: 55 = 9$} \\
\end{align*}
}

\noindent \textbf{\#19}
\sol{ HELLO
\begin{align*}
    & \text{$C=M^e \: mod \: pq$ $e = 3$ and $pq = 55$} \\
    & \text{08 05 12 12 15} \\
    & \text{$C = 08^3 \: mod 55 \:  = 512 \: mod \: 55 = 17$} \\
    & \text{$C = 05^3 \: mod 55 \:  = 125 \: mod \: 55 = 15$} \\
    & \text{$C = 12^3 \: mod 55 \:  = 1728 \: mod \: 55 = 23$} \\
    & \text{$C = 12^3 \: mod 55 \:  = 1728 \: mod \: 55 = 23$} \\
    & \text{$C = 15^3 \: mod 55 \:  = 3375 \: mod \: 55 = 20$} \\
    & \text{17 15 23 23 20 = QOWWT} \\
\end{align*}
}

\noindent \textbf{\#22}
\sol{ 13 20 20 09
\begin{align*}
    & \text{$M = C^d mod pq$ with $d = 27$ and $pq = 55$} \\
    & \text{$C = 13^{27} \: mod \: 55 = 7$} \\
    & \text{$C = 20^{27} \: mod \: 55 = 15$} \\
    & \text{$C = 20^{27} \: mod \: 55 = 15$} \\
    & \text{$C = 09^{27} \: mod \: 55 = 4$} \\
    & \text{07 15 15 04 = GOOD} \\
\end{align*}
}

\noindent \textbf{\#26}
\sol{ Use Euclidean algorith to find greatest common divisor of 6664 and 765. Express as linear combination of two numbers. 
\begin{align*}
    & \text{$6664 = 8 \cdot 765 + 544$} \\
    & \text{$765 = 1 \cdot 544 + 221$} \\
    & \text{$554 = 2 \cdot 221 + 102$} \\
    & \text{$221 = 2 \cdot 102 + 17$} \\
    & \text{$102 = 6 \cdot 17 + 0$} \\
    & \text{gcd(6664, 765) = 17} \\
    & \text{$17 = 221 - 2 \cdot 102$} \\
    & \text{$17 = 221 - 2(544 - 2(221))$} \\
    & \text{$17 = 5(221) - 2(544)$} \\
    & \text{$17 = 5(765 - 544) - 2(544)$} \\
    & \text{$17 = 5(765) - 7(544)$} \\
    & \text{$17 = 5(765) - 7(6664 - 8(765))$} \\
    & \text{$17 = 61 \cdot 765 - 7 \cdot 6664 $} \\
\end{align*}
}

\noindent \textbf{\#31a}
\sol{ Find an inverse for 210 modulo 13
\begin{align*}
    & \text{$210 = 16 \cdot 13 + 2$} \\
    & \text{$13 = 6 \cdot 2 + 1$} \\
    & \text{$2 = 2 \cdot 1 + 0$} \\
    & \text{gcd(210, 13) = 1} \\
    & \text{$1 = 13 - 6 \cdot 2$} \\
    & \text{$1 = 13 - 6(210 - 16 \cdot 13)$} \\
    & \text{$1 = 97 \cdot 13 - 6 \cdot 210$} \\
    & \text{$((-6)\cdot 210) \: mod \: 13 = (1-97 \cdot 13) \: mod \: 13$} \\
    & \text{$((-6)\cdot 210) \: mod \: 13 = 1$} \\
    & \text{Therefore, the inverse of 210 modulo 13 is -6} \\
\end{align*}
}
\noindent \textbf{\#31b}
\sol{ Find a positive inverse for 210 modulo 13
\begin{align*}
    & \text{$-6\: mod \: 13 = (-6 + 0) \: mod \: 13$} \\
    & \text{$(-6\: mod \: 13 + 13 \: mod \: 13) \: mod \: 13$ \tcp*{$0 \: mod \: 13 = 0 =13$}} \\
    & \text{$(-6 + 13) \: mod \: 13$} \\
    & \text{$7 \: mod \: 13$} \\
    & \text{Therefore, the positive inverse of 210 modulo 13 is 7} \\
\end{align*}
}
\noindent \textbf{\#31c}
\sol{ Find a positive solution for the congruence $210x \equiv 8 \pmod{13}$
\begin{align*}
    & \text{$a \equiv b \pmod{c}$ is equivalent with $a\: mod \:c = b \: mod \: c$} \\
    & \text{$210x \equiv 8 \pmod{13}$} \\
    & \text{$210x \: mod \: 13 = 8 \: mod \: 13$} \\
    & \text{$x \: mod \: 13 = 7 \cdot 8 \: mod \: 13$} \\
    & \text{$x \: mod \: 13 = 56 \: mod \: 13$} \\
    & \text{$x \: mod \: 13 = 4$} \\
    & \text{Therefore, the positive solution for the congruence $210x \equiv 8 \pmod{13}$ is 4} \\
\end{align*}
}

\noindent \textbf{\#36}
\sol{ HELP, $n = 713 = 23 \cdot 31$ and $e=43$
\begin{align*}
    & \text{$C = M^e \: mod \: pq$} \\
    & \text{H is $8^43 \: mod \: 713 = 233$} \\
    & \text{E is $5^43 \: mod \: 713 = 129$} \\
    & \text{L is $12^43 \: mod \: 713 = 048$} \\
    & \text{P is $16^43 \: mod \: 713 = 128$} \\
\end{align*}
}
\newpage

\noindent \textbf{\#39}
\sol{$n = 713 = 23 \cdot 31$ and $e=43$ and $d=307$ the inverse of 43 where $d \equiv e^{-1} \pmod{\phi(n)}$ where $n=pq$ and $\phi(n)=(p-1)(q-1)$
\begin{align*}
    & \text{$675 089 089 048$} \\
    & \text{$675^{307} \: mod \: 713 = 3$} \\
    & \text{$089^{307} \: mod \: 713 = 15$} \\
    & \text{$089^{307} \: mod \: 713 = 15$} \\
    & \text{$048^{307} \: mod \: 713 = 12$} \\
    & \text{The message is COOL} \\
\end{align*}
}

\end{document}